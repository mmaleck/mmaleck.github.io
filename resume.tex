% Copyright 2013 Christophe-Marie Duquesne <chmd@chmd.fr>
% Copyright 2014 Mark Szepieniec <http://github.com/mszep>
% Copyright 2017 Molly Maleckar <http://github.com/mmaleck>
% 
% ConText style for making a resume with pandoc. Inspired by moderncv.
% 
% This CSS document is delivered to you under the CC BY-SA 3.0 License.
% https://creativecommons.org/licenses/by-sa/3.0/deed.en_US


\startmode[*mkii]
  \enableregime[utf-8]  
  \setupcolors[state=start]
\stopmode

\setupcolor[hex]
\definecolor[titlegrey][h=2A2A2A]
\definecolor[sectioncolor][h=397249]
\definecolor[rulecolor][h=013D17]

% Enable hyperlinks
\setupinteraction[state=start, color=sectioncolor]

\setuppapersize [A4][A4]
\setuplayout    [width=middle, height=middle,
                 backspace=20mm, cutspace=0mm,
                 topspace=10mm, bottomspace=10mm,
                 header=0mm, footer=0mm]

%\setuppagenumbering[location={footer,center}]

\setupbodyfont[9pt, times]

\setupwhitespace[medium]

\setupblackrules[width=22mm, color=rulecolor]

\setuphead[chapter]      [style=\tfd]
\setuphead[section]      [style=\tfd\bf, color=titlegrey, align=center]
\setuphead[subsection]   [style=\tfb\bf, color=sectioncolor, align=right,
                          before={\leavevmode\blackrule\hspace}]
\setuphead[subsubsection][style=\bf]

\setuphead[chapter, section, subsection, subsubsection][number=no]

%\setupdescriptions[width=10mm]

\definedescription
  [description]
  [headstyle=bold, style=normal,
   location=hanging, width=12mm, distance=10mm, margin=0cm]

\setupitemize[autointro, packed]    % prevent orphan list intro
\setupitemize[indentnext=no]

\setupfloat[figure][default={here,nonumber}]
\setupfloat[table][default={here,nonumber}]

\setuptables[textwidth=max, HL=none]

\setupthinrules[width=15em] % width of horizontal rules

\setupdelimitedtext
  [blockquote]
  [before={\setupalign[middle]},
   indentnext=no,
  ]


\starttext

\section[mary-m.-molly-maleckar]{Mary M. (Molly) Maleckar}

\thinrule

\startblockquote
615 Westlake Ave N, Seattle, WA, 98109\crlf
(+1) 206 548 8449 \letterbar{}
\useURL[url1][mailto:mollym@alleninstitute.org][][Email]\from[url1]
\letterbar{}
\useURL[url2][https://www.linkedin.com/in/maleckar/][][LinkedIn]\from[url2]
\letterbar{}
\useURL[url3][https://twitter.com/mmaleck1][][Twitter]\from[url3]
\stopblockquote

\thinrule

\subsection[education]{Education}

\startdescription{2008}
  {\bf Ph.D., Biomedical Engineering}; The Johns Hopkins University
  (Baltimore, MD)
\stopdescription

\startdescription{2002}
  {\bf B.Sc., Biomedical Engineering}; Tulane University (New Orleans,
  LA)
\stopdescription

\subsection[professional-experience]{Professional Experience}

\startdescription{2017 --}
  {\bf Director}, Models and Theory, The Allen Institute for Cell
  Science, Seattle, WA.
\stopdescription

\startdescription{2015 -- 2016}
  {\bf Senior Scientist}, Computational Cardiac Modeling, Simula
  Research Laboratory, Oslo, Norway

  {\bf Training Coordinator}, AFib-TrainNet EU MCSA ITN

  {\bf Coordinator}, SysAFib ERA CoSysMed European Project
\stopdescription

\startdescription{2012 -- 2016}
  {\bf Board Member}, Simula Research Laboratory, Center for
  Cardiological Innovation (SFI) Board of Directors
\stopdescription

\startdescription{2012 -- 2015}
  {\bf Director}, Simula School of Research and Innovation, Oslo, Norway
\stopdescription

\startdescription{2011 -- 2012}
  {\bf Deputy Director for Simulation and Modeling}, Center for
  Cardiological Innovation (CCI), a Norwegian Center for Research
  Innovation (SFI)
\stopdescription

\startdescription{2011 -- 2012}
  {\bf Research Department Head}, Computational Cardiac Modeling, Simula
  Research Laboratory, Oslo, Norway
\stopdescription

\startdescription{2010}
  {\bf Research Group Leader}, Computational Cardiac Modeling and Center
  for Biomedical Computing, Simula Research Laboratory, Oslo, Norway
\stopdescription

\startdescription{2009}
  {\bf Postdoctoral Research Fellow}, Center for Biomedical Computing,
  Scientific Computing, Simula School of Research and Innovation, Oslo,
  Norway
\stopdescription

\subsection[grants-and-fellowships]{Grants and Fellowships}

\startdescription{2016 -- 2019}
  SysAFib: Systems medicine for diagnosis and stratification of atrial
  fibrillation

  ERA CoSysMed, European Commission and BIOTEK2021 Research Council of
  Norway
\stopdescription

\startdescription{2015 -- 2019}
  EU Training Network on Novel Targets and Methods in Atrial
  Fibrillation (AFib-TrainNet), Marie Skłodowska-Curie Actions,
  European\crlf
  Commission
\stopdescription

\startdescription{2015 -- 2018}
  \quotation{Risk factors for sudden cardiac death during acute
  myocardial infarction (MI-RISK)}, Novo Nordisk Foundation
  Interdisciplinary Synergy Grant
\stopdescription

\startdescription{2014 -- 2015}
  PREPARE2: Increased science awareness among youth, Simula School of
  Research and Innovation, PROFORSK, Research Council of Norway
\stopdescription

\startdescription{2014}
  Expert Advisor Policy Fellowship, The Research Council of Norway
  Brussels Office
\stopdescription

\startdescription{2012 -- 2013}
  Can Simulation shed light on a complex disease process?, Simula
  Research Laboratory/University of California San Diego, IS-BILAT,\crlf
  Research Council of Norway
\stopdescription

\startdescription{2011 -- 2019}
  The Center for Cardiological Innovation, Simula Research Laboratory,
  SFI Program, Research Council of Norway
\stopdescription

\subsection[pre-print-or-under-peer-review]{Pre-print or Under Peer
Review}

Building a 3D Integrated Cell. Gregory R. Johnson, Rory M.
Donovan-Maiye, {\bf Mary M. Maleckar}. Posted December 21, 2017. bioRxiv
238378; doi: \useURL[url4][https://doi.org/10.1101/238378]\from[url4]

Three dimensional cross-modal image inference: label-free methods for
subcellular structure prediction. Chek Ounkomol, Daniel A. Fernandes,
Sharmishtaa Seshamani, {\bf Mary M. Maleckar}, Forrest Collman, Gregory
R. Johnson. Posted November 9, 2017. bioRxiv 216606; doi:
\useURL[url5][https://doi.org/10.1101/216606]\from[url5]

Generative Modeling with Conditional Autoencoders: Building an
Integrated Cell. Gregory R. Johnson, Rory M. Donovan-Maiye, {\bf Mary M.
Maleckar}. Submitted on 28 Apr 2017.
\useURL[url6][http://arxiv.org/abs/1705.00092v1]\from[url6]

\subsection[peer-reviewed-journal-publications]{Peer-reviewed Journal
Publications}

\startdescription{2018}
  Ounkomol C, Seshamani S, {\bf Maleckar MM}, Collman F, Johnson GR.
  Label-free prediction of three-dimensional fluorescence images from
  transmitted-light microscopy. Nat Methods. 2018 Sep 17. doi:
  10.1038/s41592-018-0111-2. {[}Epub ahead of print{]}

  {\bf Maleckar MM}, Clark RB, Votta B, Giles WR. The Resting Potential
  and K(+) Currents in Primary Human Articular Chondrocytes. Front
  Physiol. 2018 Sep 4;9:974. doi: 10.3389/fphys.2018.00974. eCollection
  2018.

  Kallhovd S, {\bf Maleckar MM}, Rognes ME. Inverse estimation of
  cardiac activation times via gradient-based optimization. Int J Numer
  Method Biomed Eng. 2018 Feb;34(2). doi: 10.1002/cnm.2919. Epub 2017
  Aug 23.
\stopdescription

\startdescription{2017}
  Vagos MR, Arevalo H, de Oliveira BL, Sundnes J, {\bf Maleckar MM}. A
  computational framework for testing arrhythmia marker sensitivities to
  model parameters in functionally calibrated populations of atrial
  cells. Chaos. 2017 Sep;27(9):093941.

  Behdadfar S, Navarro L, Sundnes J, {\bf Maleckar MM}, Avril S.
  Importance of material parameters and strain energy function on the
  wall stresses in the left ventricle. Comput Methods Biomech Biomed
  Engin. 2017 Aug;20(11):1223-1232. doi: 10.1080/10255842.2017.1347160.
  Epub 2017 Jul 4.

  {\bf Maleckar MM}, Edwards AG, Louch WE, Lines GT. Studying dyadic
  structure-function relationships: a review of current modeling
  approaches and new insights into Ca(2+) (mis)handling. Clin Med
  Insights Cardiol. 2017 Apr 12;11:1179546817698602.

  Belardinelli L, {\bf Maleckar MM}, Giles WR. Ventricular Microanatomy,
  Arrhythmias, and the Electrochemical Driving Force for Na(+): Is There
  a Need for Flipped Learning? Circ Arrhythm Electrophysiol. 2017
  Feb;10(2):e004955. doi: 10.1161/CIRCEP.117.004955. Erratum in: Circ
  Arrhythm Electrophysiol. 2017 Mar;10 (3):.
\stopdescription

\startdescription{2016}
  Skibsbye L, Jespersen T, Christ T, {\bf Maleckar MM}, van den Brink J,
  Tavi P, Koivumäki JT. Refractoriness in human atria: Time and voltage
  dependence of sodium channel availability. J Mol Cell Cardiol. 2016
  Dec;101:26-34.

  Grandi E, {\bf Maleckar MM}. Anti-arrhythmic strategies for atrial
  fibrillation: The role of computational modeling in discovery,
  development, and optimization. Pharmacol Ther. 2016 Sep 6. pii:
  S0163-7258(16)30168-1.

  Lines GT, Oliveira BL, Skavhaug O, {\bf MM Maleckar}. Simple T wave
  metrics may better predict early ischemia as compared to ST segment.
  IEEE Transactions on Biomedical Engineering, 2017 Jun;64(6):1305-1309.
  doi: 10.1109/TBME.2016.2600198. Epub 2016 Aug 25.
\stopdescription

\startdescription{2015}
  S. Kallhovd, S.U. Gerald,~J. Saberniak, K. Haugaa,~{\bf MM
  Maleckar}.Localization and not Extent of Fibrofatty Infiltration~is
  the Primary Factor Determining Conduction~Disturbance in a
  Computational Model of~Arrhythmogenic Cardiomyopathy. Proceedings IEEE
  e-Health and Bioengineering 2015. EHB 2015, November 19-21, 2015,
  Iasi, Romania.
\stopdescription

\startdescription{2014}
  {\bf Maleckar MM}, Lines GT,~Koivumäki J,~ Cordeiro JM,~Calloe
  K.NS5806 partially restores action potential duration but fails to
  ameliorate calcium transient dysfunction in a computational model of
  canine heart failure, 2014 Nov;16 Suppl 4:iv46-iv55.

  Koivumäki JT, Clark RB, Belke D, Kondo C, Fedak PW,~{\bf Maleckar MM},
  Giles WR.Na(+) current expression in human atrial myofibroblasts:
  identity and functional roles. Front Physiol. 2014 Aug 7;5:275. doi:
  10.3389/fphys.2014.00275. eCollection 2014.

  Frisk M,~Koivumaki~J, Norseng PA,~{\bf Maleckar~MM}, Sejersted OM,
  Louch WE. Variable t-tubule organization and Ca2+ homeostasis across
  the atria. Am J Physiol Heart Circ Physiol. 2014 Jun 20.~

  Koivumäki JT, Seemann G,~{\bf Maleckar~MM}, Tavi P. In silico
  screening of the key cellular remodeling targets in chronic atrial
  fibrillation. PLoS Comput Biol. 2014 May 22;10(5):e1003620. doi:
  10.1371/journal.pcbi.1003620. eCollection 2014 May.

  Yuan L, Koivumäki JT, Liang B, Lorentzen LG, Tang C, Andersen MN,
  Svendsen JH, Tfelt-Hansen J,~{\bf Maleckar~M}, Schmitt N, Olesen MS,
  Jespersen T. Investigations of the Navß1b sodium channel subunit in
  human ventricle; functional characterization of the H162P Brugada
  syndrome mutant. Am J Physiol Heart Circ Physiol. 2014 Apr
  15;306(8):H1204-12. doi: 10.1152/ajpheart.00405.2013. Epub 2014 Feb
  21.
\stopdescription

\startdescription{2013}
  Li P, Lines GT, {\bf Maleckar MM}, Tveito A. Mathematical Models of
  Cardiac Pacemaking Function. Frontiers in Physics, 1(20): 2013
  http://www.frontiersin.org/Journal/10.3389/fphy.2013.00020/abstract

  Wilhelms M, Hetmann H, {\bf Maleckar MM}, Koivumäki J, Dossel O,
  Seeman G. Benchmarking electrophysiological models of human atrial
  myocytes, Frontiers in Physiology 3(487), 2013.
\stopdescription

\startdescription{2012}
  Koivumäki J, Christ T, Seemann G, and {\bf Maleckar MM}. Divergent
  action potential morphology in human atrial cells vs.~tissue:
  underlying ionic mechanisms, In: Computing in Cardiology, ed. by Alan
  Murray, vol.~39, pp.~121-124, Alan Murray (ISBN: 978-1-4673-2076-4),
  2012. Refereed proceedings.

  Rose RA, Belke DD,~{\bf Maleckar MM}, Giles WR. Ca2+ Entry Through
  TRP-C Channels Regulates Fibroblast Biology in Chronic Atrial
  Fibrillation. Circulation 126(17): 2039-41, 2012.

  Tveito A, Lines GT, Edwards AG, {\bf Maleckar MM}, Michailova A, Hake
  J, McCulloch A. Slow Calcium-Depolarization-Calcium waves may initiate
  fast local depolarization waves in ventricular tissue. Prog Biophys
  Mol Biol 110(2-3): 295-304, 2012.

  Tveito A, Lines G, Rognes ME, and {\bf Maleckar MM}. An analysis of
  the shock strength needed to achieve defibrillation in a simplified
  mathematical model of cardiac tissue. International Journal of
  Numerical Analysis and Modeling 9(3): 644-57, 2012.

  Tveito A,~Lines G,~and {\bf Maleckar MM}.~ Note on a possible
  pro-arrhythmic property of anti-arrhythmic drugs aimed at improving
  gap-junction coupling. Biophys J 102(2): 231-37, 2012.
\stopdescription

\startdescription{2011}
  Niederer SA, Kerfoot E, Benson A, Bernabeu MO, Bernus O, Bradley C,
  Cherry EM, Clayton R, Fenton FH, Garny A, Heidenreich E, Land S,
  {\bf Maleckar M}, Pathmanathan P, Plank G, Rodríguez JF, Roy I, Sachse
  FB, Seemann G, Skavhaug O and Smith NP. N-Version Benchmark Evaluation
  of Cardiac Tissue Electrophysiology Simulators. Philosophical
  Transactions of the Royal Society VPH Issue. Philos Transact A Math
  Phys Eng Sci. 369(1954): 4331-51, 2011.

  McDowell K, Arevalo H, {\bf Maleckar MM}, and Trayanova NA.
  Susceptibility to reentry in the infarcted heart depends on active
  fibroblast density. Biophysical Journal 101(6): 1307-15, 2011.

  Tveito A, Lines G, Skavhaug O, and {\bf Maleckar MM}. Unstable
  eigenmodes are possible drivers for cardiac arrhythmias. Journal of
  the Royal Society Interface. 8(61): 1212-6, 2011.~

  Tveito A, Lines G, Artebrant R, Skavhaug O, and {\bf Maleckar MM}.
  Existence of excitation waves for a collection of cardiomyocytes
  electrically coupled to fibroblasts. Mathematical Biosciences 230(2):
  79-86, 2011.
\stopdescription

\startdescription{2009 --}
  {\bf Maleckar MM}, Greenstein JL, Giles WR, and Trayanova NA.
  Electrotonic coupling between human atrial myocytes and fibroblasts
  alters excitability and repolarization. Biophysical Journal 97(8):
  2179-2190, 2009.

  {\bf Maleckar MM}, Greenstein JL, Giles WR, and Trayanova NA.
  Repolarization in the human atrial myocyte -- rate-dependent changes
  in the action potential waveform. Am J Physiol Heart Circ Physiol
  297(4): 1398-1410, 2009.

  {\bf Maleckar MM}, Greenstein JL, Trayanova NA, and Giles WR.
  Mathematical simulations of ligand-gated and specific cell-type
  effects in the human atrium. Prog Biophys Mol Biol 98: 161-70, 2008.

  {\bf Maleckar MM}, Woods MC, Sidorov VY, Holcomb MR, Mashburn DN,
  Wikswo JP and Trayanova NA. Polarity reversal lowers activation time
  during diastolic field stimulation of the rabbit ventricles: Insight
  into mechanisms. Am J Physiol Heart Circ Physiol 295(4):H1626-33,
  2008.

  Bourn DW, {\bf Maleckar MM}, Rodríguez B, Trayanova NA. Mechanistic
  enquiry into the effect of increased pacing rate on the upper limit of
  vulnerability. Phil Trans. Royal Soc A, 346:1333-1348, 2006.

  Gurev V, {\bf Maleckar MM}, and Trayanova NA. Cardiac Defibrillation
  and the Role of Mechano-Electric Feedback in Postshock
  Arrhythmogenesis. The Annals of the New York Academy of
  Sciences,1080:320-333, 2006.
\stopdescription

\subsection[selected-conferences-and-talks]{Selected Conferences and
Talks}

{\em Maleckar MM}. Stem cell organization using label-free imaging and a
novel generative model. Biophysical Society Annual Meeting Computational
Biology Platform. February 18th, 2018. BPS 2018, San Francisco, CA.
Invited talk.

{\em Maleckar MM}. Capturing variance: integrating a moving target.
Building the Cell 2017 Subgroup Meeting, December 2nd, 2017, ASCB,
Philadelphia, PA. Invited talk.

{\em Maleckar MM}. Putting the pieces together: Towards supplementing
sparse clinical data with multi physics simulation Foundation Teofilo
Rossi di Montelera Forum 2015, December 6-9, 2015, Lugano, Switzerland.
Invited talk.

{\em Maleckar MM}. How many ionic models do we need for modelling of the
atria? Atrial Signals 2015, Karlsruhe, Germany, 22.-24. October. Invited
talk.

{\em Maleckar MM}. Patient-specific modeling: how good do we have to be?
TRM Forum 2013, December 1-3, 2013, Lugano, Switzerland. Invited talk.

{\em Maleckar MM}, Lines GT, Koivumäki JT, Calloe K, Cordeiro JM.
Ca2+-transient dysfunction and ion channel therapy: what can we gather
from a computational model of canine heart failure? EHRA Scientific
Sessions 2013, 37th Annual Meeting of the ESC Working Group on Cardiac
Cellular Electrophysiology, 2013. Poster and presentation.

{\em Maleckar MM}. Towards Modeling Arrhythmogenic Cardiomyopathy -- Can
Simulation Shed Light on a Complex Disease Process? Cardiac Physiome
Workshop, San Diego, October 30 -- November 2, 2012. Invited talk.

{\em Maleckar MM}. Modeling the effects of rotigaptide in atrial tissue:
a cautionary tale. 9th International Conference of Numerical Analysis
and Applied Mathematics, September 19-25, 2011. Invited talk.

\subsection[professional-affiliations]{Professional Affiliations}

\startitemize[packed]
\item
  2016 -- present, Member, American Society for Cell Biology
\item
  2011 -- present, Member, Scandinavian Physiological Society
\item
  2011 -- present, Member, Biophysical Society
\item
  2011 -- present, Member, European Society of Cardiology Working Group
  on Cellular Cardiac Electrophysiology
\item
  2012 -- present, Member, European Society of Cardiology Working Group
  on eCardiology
\item
  2009 -- present, Member, Heart Rhythm Society
\item
  2009 -- present, Member, American Association for the Advancement of
  Science
\stopitemize

\subsection[other]{Other}

\startitemize[packed]
\item
  English (native); Spanish (C2); Norwegian (B2)
\item
  10+ years consultation experience in science communication, including
  presentation, popular, and technical writing.
\stopitemize

\thinrule

\startblockquote
\useURL[url7][mailto:mollym@alleninstitute.org][][mollym@alleninstitute.org]\from[url7]
• +1 (206) 548 8449 • 615 Westlake Ave N, Seattle, WA, 98107
\stopblockquote

\stoptext
